% Based on free LaTex template https://www.overleaf.com/latex/templates/modelo-tcc-computacao-atitus/dgwsczcmpczz
\documentclass[14pt]{article}

\usepackage{sbc-template}
\usepackage{graphicx, url}
\usepackage[utf8x]{inputenc}
\usepackage[T1]{fontenc}
\usepackage[english=american]{csquotes}
\usepackage{float}
\usepackage{comment}
\usepackage{amsmath}
\usepackage{amssymb}
\usepackage{enumerate}
\usepackage{subcaption}
\usepackage{setspace}
\usepackage{listings}
\usepackage{inconsolata}
\usepackage{tabularray}
\usepackage[english, russian]{babel}

%\usepackage[backref=page]{hyperref}
%\hypersetup{
%    colorlinks=true,
%    allcolors=blue,
%}

\usepackage[style=abnt]{biblatex}
\addbibresource{sbc-template.bib}

\usepackage{sty/cc_atitus}


\title{Раскрашивальщик}

\author{Бабанский Виталий, Бакин Денис}

\address{}


\begin{document}

\maketitle


\section{Постановка задачи}

Задача восстановления цветных изображений из черно-белых снимков является распространенной задачи и применяется, например,
при обновлении исторических снимков, которые были сделаны до изобретения цветной фотографии.
Более сложной постановкой той же задачи считается раскрашивание снимков NIR (near-infrared spectroscopy) ---
это снимки, где вместо количества видимого света фотосенсором камеры подсчитывается количестве фотонов с длиной волны
от 780 нм до 2500 нм, то есть выше видимого диапазона. Такая съемка применяется при низкой освещенности и
при съемке архитектурных объектов.

\begin{figure}[H]
    \centering
    \includegraphics[width=0.5\textwidth]{resources/pisa_tower_3_colorspaces.jpg}
    \caption{Три пространства цветов: RGB, черно-белое и NIR}
    \label{fig:id_figura}
\end{figure}


\section{Постановка задачи}
Целью проекта является создание и обучение нейронной сети для получения цветных изображений по данным черно-белым изображениям,
а также провести ряд эксперименты, воспроизвести результаты выбранных статей и измерить полученное качество по набору метрик.


\section{Литература}
Список рассмотренных не окончательный. Включены только те статьи, идеи которых скорее всего будут использовать в реализации.

\subsection{Раскрашивание с подсказками}
\cite{GuidedImageColorization} предлагает архитектуру полносверточной нейронной сети, которая принимает на вход
черно-белое изображение и набор локальных и глобальных подсказок от пользователя. Сеть раскрашивает указанные пиксели так, как
скажет пользователь, а остальное изображение так, чтобы оно было наиболее естественным. Сеть показыват примлемое качество
как бейзлайн: основная полносверточная нейросеть используется в некоторых других более сложных архитектурах. Отличная качество достигается
в особо сложных случаях, когда на снимке есть мелкий арнамент или цвета, которые сложно восстановить из контекста (воздушные шары, например).

\subsection{Instance colorization}
\cite{InstaColor} использует основную полносверточную нейронную сеть, из \cite{GuidedImageColorization}. Идея авторов заключается в генерации
ограничивающих прямоугольников (bounding boxes) вокруг известных объектов на изображении с помощью предтренированного детектора \cite{MaskCNN}.
Затем с помощью выбранного backbone раскрашиваются как вырезанные объекты, так и все изображение в целом. Затем на этапе карт признаков
модуль слияния "мягко" объединяет вырезанные раскрашенные объекты и полное раскрашенное изображение. Это дает улучшенные результаты по сравнению с прошлыми статьями
и относится к полностью автоматическому раскрашиванию черно-белых снимков.

\subsection{Cooperative colorization}
\cite{CoColor} авторы решают целых 2 проблемы: улучшают качество раскрашивания и предлагаются объединение и трансфер знаний модели между двумя доменами входных данных:
черно-белых и NIR изображений. В статье предлагается генерировать альтернативный домен по данному (NIR по черно-белому изображению или наоборот). Затем каждое
из изображений раскрашивается, результат объединяется. Поскольку такое количество генеративных сетей может отлоняться от ответа, авторы статьи предлагают множество дополнительных
ограничений для модели в виде функций потерь, которые требуют, чтобы раскрашенные изображения из разных доменов были очень похожи по структуре (ведь цвет ее не меняется).

\subsection{NIR-to-RGB Spectral Translation with Mamba}
\cite{ColorMamba} является лучшей на данный момент архитектурой для раскрашивания NIR изображений (на датасете NIR изображений с подготовленной валидационной выборкой
\cite{VCIP_2020_NIR_dataset}). Основа подхода заключается в построении двух набора связанных модулей: сети для раскрашивания в простанство RGB,
сети для раскрашивания в простанство HSV, а также набора более компактных неглубоких подмодулей, описанных авторами статьи.

\begin{figure}[H]
    \centering
    \includegraphics[width=1.05\textwidth]{resources/performance.pdf}
    \caption{Результаты работы ColorMambda \cite{ColorMamba}. Также приведено сравнение с CoColor \cite{CoColor}}
    \label{fig:id_figura}
\end{figure}


\section{Данные}
Авторы статей \cite{GuidedImageColorization}, \cite{InstaColor}, \cite{CoColor} использовали датасеты COCO \cite{COCO} и ImageNet \cite{ImageNet}.
Авторы статьи \cite{ColorMamba} использовали датасет NIR изображений с подготовленной валидационной выборкой \cite{VCIP_2020_NIR_dataset}.
Мы планируем использовать выборки из датасетов ImageNet, который содержит сфокусированные фотографии различных объектов, и COCO, который
содержит более общие сцены: архитектуры, природы.

Возможно, будут проведены эксперименты с созданием генерации черно-белого изображения по данному NIR данным. В этом случае к данным будет добавлен датасет
"RGB-NIR Scene Dataset".


\section{Метрики качества}
Для оценки качества раскрашивания снимков будем использовать набор метрик. По ним же будем сравнивать качество работы моделей.

\begin{itemize}
    \item \textbf{MSE}. Один из наиболее очевидных методов оценки близости предсказания к "верному" ответу.
    К недостаткам этой метрики можно отнести неразличимость мелкой зашумленности и отсутствия контроля за резкими переходами цветов,
    которые требуются при корректном раскрашивании изображений.

    \item \textbf{PSNR (Пиковое отношение сигнал/шум)}: PSNR измеряет качество цветных изображений, сравнивая пиксельные различия между оригиналом и 
    раскрашенным изображением. Более высокие значения PSNR указывают на лучшее качество изображения с меньшими искажениями.

    \item \textbf{SSIM (Индекс структурного сходства)}: SSIM оценивает структурное сходство между оригиналом и раскрашенным изображением, 
    учитывая яркость, контрастность и текстуру. Этот индекс предоставляет более точную для восприятия меру качества изображения по сравнению 
    с метриками на основе пикселей, такими как PSNR.

    \item \textbf{AE (Абсолютная ошибка)}: AE количественно оценивает абсолютное различие между соответствующими пикселями оригинала и 
    раскрашенного изображения. Меньшие значения AE указывают на лучшую точность раскраски.

    \item \textbf{LPIPS (Обученное перцептуальное сходство изображений)} \cite{PerceptualMetric}: LPIPS оценивает перцептуальное сходство с использованием моделей 
    глубокого обучения, фокусируясь на том, как человеческое зрение воспринимает различия между оригиналом и раскрашенным изображением. 
    Более низкие значения LPIPS означают, что раскрашенным изображение более точно соответствует восприятию человека.
\end{itemize}


\section{Текущая идея}
Идея на момент написания отчета и вероятно изменится в будущем.

В качестве бейзлайна хочется реализовать полносверточную нейронную сеть из \cite{GuidedImageColorization} и провести ряд экспериментов, возможно,
с реализацией интерактивных подсказок в пользовательском интерфейсе. Затем хочетлось бы реализовать одну из других рассмотренных статей и проверить
воспроизводимость результатов по выбранным метрикам.



\printbibliography

\end{document}
